

 % !TEX encoding = UTF-8 Unicode

\documentclass[a4paper]{report}

\usepackage[T2A]{fontenc} % enable Cyrillic fonts
\usepackage[utf8x,utf8]{inputenc} % make weird characters work
\usepackage[serbian]{babel}
%\usepackage[english,serbianc]{babel}
\usepackage{amssymb}

\usepackage{color}
\usepackage{url}
\usepackage[unicode]{hyperref}
\hypersetup{colorlinks,citecolor=green,filecolor=green,linkcolor=blue,urlcolor=blue}

\newcommand{\odgovor}[1]{\textcolor{blue}{#1}}

\begin{document}

\title{Analiza izazova i olakšica za studente prve godine Informatike na  Matematičkom fakultetu\\ \small{Stojanović Nikola, Vuk Stefanović, Kastratović Vladimir, Viktor Nikić}}

\maketitle

\tableofcontents

\chapter{Recenzent \odgovor{--- ocena: 4} }


\section{O čemu rad govori?}
% Напишете један кратак пасус у којим ћете својим речима препричати суштину рада (и тиме показати да сте рад пажљиво прочитали и разумели). Обим од 200 до 400 карактера.

Rad "Šta je najlakše studentima prve godine?" istražuje faktore koji pomažu studentima prve godine osnovnih studija da se lakše prilagode akademskom obrazovanju i daje predloge za unapređenje metodoloških pristupa u nastavi. Kroz anketu sprovedenu među 102 studenta Informatike na Matematičkom fakultetu, analizirani su faktori poput predznanja, dostupnosti informacija, korisnosti konsultacija i metoda praćenja gradiva. Na osnovu rezultata ankete i relevantne literature, autori predlažu konkretne mere za poboljšanje obrazovanja, uključujući pripremne kurseve, mentorski program, interaktivnu nastavu i inovativne pristupe poput "flip" nastave i "gamifikacije", kako bi studenti lakše savladali izazove i postigli bolje rezultate.


\section{Krupne primedbe i sugestije}
% Напишете своја запажања и конструктивне идеје шта у раду недостаје и шта би требало да се промени-измени-дода-одузме да би рад био квалитетнији. 

Uvod sadrži previše detalja i specifičnih informacija koje bi bolje odgovarale u kasnijim delovima rada. Konkretno, deo:

\textit{"Prema istraživanjima, studenti često identifikuju određene aspekte studiranja koji im deluju lakše, kao što su određeni kursevi ili predmeti koji im omogućavaju lakšu integraciju u akademski sistem..."} (do kraja drugog pasusa)

sadrži detalje i preporuke koji se tiču strategija podrške studentima, kao što su organizovanje tutorijala, dodatnih vežbi i upravljanje vremenom. Ovaj deo bi bolje odgovarao u delu koji se bavi razradom teme, nego u uvodu. 
\newline

Ako je cilj ovog dela da se objasni izbor pitanja za anketu, preporučujem da se tekst sažme kako bi uvod bio kraći, jasniji i usmereniji na glavne ciljeve istraživanja. Na taj način, uvod bi postao jasniji i bolje usmeren ka definisanju osnovnih pitanja koja će se istraživati. \odgovor{Nakon ponovnog prelaska preko Uvoda, sa ovim primedbama u vidu, slažemo se da se taj deo teksta bolje uklapa u kasniji deo rada i ceo drugi pasus je premešten na početak drugog dela rada. Takođe je umesto njega stavljen drugi tekst koji više upućuje čitaoca u to šta sledi.}



\section{Sitne primedbe}
% Напишете своја запажања на тему штампарских-стилских-језичких грешки

Citiranje:
\begin{itemize}
    \item Citati [3], [4], [5] i [7] se nalaze iza tačke ("vise u vazduhu"), a trebalo bi da budu postavljeni pre tačke kako bi bilo jasnije na šta se konkretno odnose. Takođe, mislim da se svi citati nalaze posle tačke u rečenici na koju se odnose. \odgovor{Citati su pomereni ispred tačke}
    \item Bilo bi korisno povezati citate sa odgovarajućim izvorima u literaturi putem hiperlinkova, kako bi se olakšalo pronalaženje izvora u literaturi na kraju rada. \odgovor{Citati su povezani sa odgovarajućim izvorima u literatura preko hiperlinkova.}
\end{itemize}

Tabele:
\begin{itemize}
    \item Na nekoliko mesta tabela izlazi iz okvira rada i prelazi "{}desnu marginu". \odgovor{Tabele su prepravljene i poravnjate.}
    \item Nema konzistentnosti u korišćenju bolda i italika u tekstu koji se odnosi na tabele: na primer, „Zaključak na osnovu Tabele 2, 3, ...“ se ponekad piše sa boldovanjem (\textbf{Zaključak na osnovu Tabele ... :}), a ponekad sa italikom (\textbf{Zaključak} na osnovu \textit{Tabele ...}), bilo bi bolje ostati dosledan jednom formatu. \odgovor{Sav tekst oblika  „Zaključak na osnovu Tabele ...“ je sada boldiran. Posle detaljnijeg prelaženja odlučeno je da ceo tekst tog oblika bude izbačen i sada su samo imena tabela unutar teksta boldirani.}
    \item Od Tabele 7, postoji neslaganje u numeraciji između tabele i zaključka. \odgovor{Ispravljene su greške u numeraciji.}
\end{itemize}

Literatura:
\begin{itemize}
    \item Ne radi prvi link u Literaturi, tj otvara se stranica na kojoj piše: DOI Not Found. \odgovor{Postavljen je ispravan link.}
\end{itemize}




\section{Provera sadržajnosti i forme seminarskog rada}
% Oдговорите на следећа питања --- уз сваки одговор дати и образложење

\begin{enumerate}
\item Da li rad dobro odgovara na zadatu temu?\\
Da, rad odgovara na zadatu temu i fokusira se na faktore koji olakšavaju prilagođavanje studenata prve godine.

\item Da li je nešto važno propušteno?\\
Ne mislim da je nešto važno propušteno u radu.

\item Da li ima suštinskih grešaka i propusta?\\
Ne mislim da ima suštinskih grešaka i propusta. Lepo je pokrivena suština teme.

\item Da li je naslov rada dobro izabran?\\
Naslov je jasan i adekvatan za temu rada, opisuje suštinu rada i daje preciznu sliku o tome šta ćemo saznati u radu.


\item Da li sažetak sadrži prave podatke o radu?\\
Da, sažetak pruža dobar pregled rada i ono šta možemo očekivati u radu.

\item Da li je rad lak-težak za čitanje?\\
Rad je lak za čitanje, stil je jasan i razumljiv.

\item Da li je za razumevanje teksta potrebno predznanje i u kolikoj meri?\\
Ne, nije bilo potrebno nikakvo predznanje. Rad se bavi pitanjima koja se odnose na adaptaciju studenata na akademsko obrazovanje, što je relevantno za sve čitaoce, bez obzira na prethodno znanje.

\item Da li je u radu navedena odgovarajuća literatura?\\
Da, literatura je odgovarajuća. U radu su korišćeni izvori koji uključuju knjige, naučne članke i relevantne veb adrese.

\item Da li su u radu reference korektno navedene?\\
Reference su korektne, ali bi trebalo da se postave pre tačke u rečenici na koju se odnose, kako bi se izbegla konfuzija i poboljšala čitljivost. \odgovor{Reference su premeštene kao odgovor na primedbu u delu 2.3.}

\item Da li je struktura rada adekvatna?\\
Struktura rada je adekvatna, sa svim potrebnim delovima: sažetak, uvod, razrada, zaključak i literatura. Međutim, uvod sadrži previše detalja kojim smatram da nije tu mesto. \odgovor{Uvod je izmenjen kao odgovor na primedbu u delu 2.2. }

\item Da li rad sadrži sve elemente propisane uslovom seminarskog rada (slike, tabele, broj strana...)?\\
Da, rad sadrži sliku, tabele i 7 referenci što ispunjava osnovne zahteve za seminarski rad.

\item Da li su slike i tabele funkcionalne i adekvatne?\\
Slike i tabele su funkcionalne i adekvatno postavljene.

\end{enumerate}

\section{Ocenite sebe}
% Napišite koliko ste upućeni u oblast koju recenzirate: 
% a) ekspert u datoj oblasti
% b) veoma upućeni u oblast
% c) srednje upućeni
% d) malo upućeni 
% e) skoro neupućeni
% f) potpuno neupućeni
% Obrazložite svoju odluku

Srednje sam upućena u ovu temu. Imam osnovno razumevanje i, iako nisam zalazila u sve detalje, moje prethodno iskustvo u vezi sa temom mi je pomoglo da dam, nadam se, korisne sugestije.


\chapter{Recenzent \odgovor{--- ocena: 5} }


\section{O čemu rad govori?}
% Напишете један кратак пасус у којим ћете својим речима препричати суштину рада (и тиме показати да сте рад пажљиво прочитали и разумели). Обим од 200 до 400 карактера.
Rad se bavi prilagođavanjem studenata prve godine na univerzitetsko obrazovanje, istražujući koje aspekte studiranja doživljavaju kao lakše ili teže. Kroz analizu anketa i dobijenih rezultata, rad identifikuje ključne izazove sa kojima se studenti suočavaju i nudi predloge za njihovo prevazilaženje, uključujući unapređenje nastavnih metoda, podrške i organizacije vremena.

\section{Krupne primedbe i sugestije}
% Напишете своја запажања и конструктивне идеје шта у раду недостаје и шта би требало да се промени-измени-дода-одузме да би рад био квалитетнији.

Mislim da se sažetak treba korigovati. Samo poslednja rečenica govori konkretno o samom radu. Izbacio bih deo ispred i dodao još neku rečenicu o radu da bih privukao pažnju čitalaca i da bih im približio o čemu ste tačno istraživali u radu. \odgovor{Nakon razmatranja primedbe korigovali smo prvu rečenicu, ali nismo nista vise dodali jer smatramo da sažetak već dovoljno govori o tome šta se tačno istražuje u radu.}

Sadržaj nije na prvoj strani sa naslovom i apstraktom. Mislim da možete to rešiti smanjivanjem fonta sadržaja ili ako promenite strukturu rada(videćete preporuku nešto kasnije) biće sve u redu. \odgovor{Sadržaj je sada premešten na prvu stranu.} \\

Uvod je predugačak i sadrži ponavljanja. Nakon čitanja nije me privukao i dao motivaciju za daljim čitanjem. Nije predstavljena jasno svrha rada, istraživačka pitanja i rezultati. Smatram da uvod treba imati jasniju strukturu sa fokusom na neke od ključnih tačaka, svrhu istraživanja i najavu nekih interesantnih nalaza što bi ga učinilo privlačnijim za čitanje. Takođe, reference u uvodu ne funkcionišu. \odgovor{Uvod je skraćen izbacivanjem drugog pasusa(koji je premešten na početak drugog dela rada). Umesto drugog pasusa dodat je kratak tekst za koji mislimo da daje solidnu najavu u to šta sledi dalje u radu. Takođe reference u uvodu su popravljene.} \\


Druga sekcija započinje paragrafom koji sadrži samo jednu rečenicu, što nije dozvoljeno. Tabela, koja prikazuje da je u istraživanju učestvovalo 102 ispitanika, čini se suvišnom. Ova informacija se lako može uključiti u tekst, čime bi se izbegla redundancija i nepotrebno zauzimanje prostora. Nakon tabele, rečenica koja sumira broj ispitanika i osnovne karakteristike istraživanja mogla bi biti dovoljna, bez potrebe za dodatnim vizualnim prikazom.Pored toga, završetak paragrafa sa "predstavljene su osnovne statistike i zaključci:" deluje nedovršeno i navodi na pogrešan utisak da sledi neko nabrajanje. Umesto toga, završetak bi trebalo jasno da signalizira prelazak na analizu podataka ili objašnjenje rezultata. Naslov tabele mora biti pravilno pozicioniran iznad tabele, bez tačke na kraju, i na tabelu se mora jasno referisati u tekstu. Nijedna od ovih stvari nije ispoštovana.
Preporučujem uklanjanje suvišne tabele, bolje strukturiranje paragrafa i poštovanje pravila za referisanje i naslov tabela. \odgovor{Na početak sekcije je stavljen drugi pasus koji je prethodno stajao u prvoj sekciji, mislimo da je on ovde prikladniji, a ujedno i dopunjava ovaj deo. Tabela je takođe uklonjena i informacije iz nje su uključene u tekst. Završetak je sada ispravljen i smatramo da jasno nagoveštava prelazak na analizu podataka.} \\

Podsekcija 2.1 je nepotrebna i ne doprinosi suštinski kvalitetu rada. Informacija o ravnopravnoj zastupljenosti polova među ispitanicima mogla je biti uključena u širi kontekst, poput uvoda u sekciju ili u drugu podsekciju gde je ta informacija relevantna za dalju analizu. 

Podsekcija sadrži samo jednu rečenicu, što nije dozvoljeno. Takođe, počinje grafikom, što je formalno nepravilno — paragraf uvek treba da započne tekstom koji pruža uvod i kontekst za grafički prikaz.

Dodatno, referenciranje grafika nije pravilno izvedeno. Grafici se u tekstu moraju jasno navesti, sa preciznim opisom, na primer: "Na osnovu slike N možemo zaključiti...". Trenutni stil, gde se analiza rezultata započinje frazom "Zaključak na osnovu slike/tabele N:", zvuči previše direktno i neformalno. Preformulisanje u pristup koji obuhvata objašnjenje i upućivanje na grafički prikaz značajno bi poboljšalo analizu.

Za ovu podsekciju, preporučujem uklanjanje zasebnog naslova i uključivanje informacije o polovima na drugom mestu, uz pravilno referenciranje grafika i bolju formulaciju zaključaka(u ovoj i svim narednim sekcijama). \odgovor{Podsekcija je uklonjena(samim tim su sve ostale numerisane drugacije tj. sada je sekcija 2,2 zapravo sekcija 2.1 itd), a informacije iz nje su dodate u zanji pasus uvodnog dela Sekcije 2.}\\

U sekciji 2.2 ste počeli paragraf tabelom (što je neispravno) na koju ponovo nije ispravno referisano, takođe naslov tabele je ispod a treba iznad i tabela izlazi iz okvira margina.
Opet koristite ovu frazu sa dvotačkom na kraju za zaključak koja mi se uopšte ne sviđa. Nju imate u svakoj od sekcija i neću više to pominjati da se ne bih previše ponavljao.\\
Ne možete samo staviti "preporuke:" i krenuti da nabrajate. Šta nabrajate? Za koga su namenjene preporuke? Sa kojim ciljem?
Morate ovo da izmenite, na primer možete dodati neku rečenicu "Neke od preporuka studentima sa malim ili nikakvim predznanjem iz programiranja kako bi se lakše prilagodili na akademski sistem su:" i onda krenete da nabrajate. \odgovor{Početak paragrafa je ispravljen i tabela je ispravno referisana i naslov je stavljen iznad tabele, takođe tabela je prepravljena da bude u okviru margina. I pored toga što je primedba o frazi sa dvotačkom, pomalo pristrasna i nije moralo toliko da se napomene da li se to recezentu svidja ili ne ipak smo to uzeli u obzir i slažemo se sa primedbom i to smo ispravili. Takođe uvažena je primedba o preporukama i ispravljeno je.} \\

U sekciji 2.3 imate sve iste greške kao i u sekciji 2.2, dakle naslov tabele, margine, počinjete paragraf tabelom, ne svidja mi se ovo za nabrajanje preporuka, niste objasnili ništa o tome za koga su preporuke, sa kojim ciljem ih dajete i tako neke stvari.
U ovoj sekciji imate još jednu dodatnu greškicu a to je ponavljanje podataka iz tabele što nije prihvatljivo osim ako ne može da se izbegne.
Vi ste već stavili broj 34,3\% u zagrade što već govori da nije neophodno da stoji tu, iz tabele se lako vidi koji procenat ispitanika glasao za to tako da možete samo izostaviti ovaj broj ali morate u tekstu pravilno referisati na tabelu. \odgovor{Iste greške koje su napomenute za podesekciju 2.2 su ispravljene i ovde, takodje podaci iz tabele se vise ne nalaze u tekstu.} \\

U sekciji 2.4 ponovljene sve greške iz oblasti 2.3 i 2.2\\
U sekciji 2.5 takođe, sa tim da bih dodao da mi je tabela malo nečitljiva zbog ova dva podatka u jednom polju. \\
U sekcijama 2.6, 2.7, 2.9, 2.10, 2.11 identične greške kao u prethodnim sekcijama.\\
Sekcija 2.8 uz sve prethodno navedene greške ja i vizuelno loša i nečitljiva jer su tabele natrpane jedna ispod druge.\odgovor{U svim navedenim sekcijama ispravljene su greške koje su primećene u sekcijama 2.3 i 2.2, a koje se i ovde ponavljaju. U sekciji 2.5(nakon izmene rada to je 2.4) mislimo da su tabele dovoljno čitljive i da ih nije potrebno menjati. U skeciji 2.8 (nakon izmena 2.7) nismo promenili nista u vezi tabela jer smatramo da je ureu ovako kako već jeste.} \\

U zaključku imate probleme sa referencama kao i u ostatku teksta. \odgovor{Problemi sa referencama su ispravljeni na nivou celog teksta.} \\

Nakon detaljnog čitanja vašeg rada predložio bih vam par par većih promena.
Prva se tiče samog naslova. U vašem radu ste više pažnje posvetili tome kako olakšati novim studentima adaptaciju na akademski sistem i na sam proces studiranja nego na to šta je najlakše studentima prve godine. Mislim da trebate promeniti naslov, dozvoljeno je koliko znam. \odgovor{Slažemo se i nakon ove primedbe promenili smo naslov rada u 'Analiza izazova i olakšica za studente prve godine Informatike na Matematičkom fakultetu'.}
Druga stvar se se odnosi na samo strukturu vašeg rada. Pretpostavljam da ste za svako pitanje sa ankete napavili podsekciju i smatram da to nije potrebno. Prvo sekcija o polu ispitanika je totalno bespotrebna zato što se nigde u daljem tekstu ne pozivate na tu informaciju, da ste negde uporedili različito glasanje žena i muškaraca pa i da razumem ali tokom celog rada analizirate rezultate cele grupe ispitanika i nema potreba da imate podsekciju o polu.
Možda trebate razmisliti o tome da imate jednu sekciju u kojoj ce vam biti analiza ankete, diskusija o tome šta je teško studentima, zbog čega i tako to a nakon toga da imate celu sekciju u kojoj ćete diskutovati koje su opcije koje bi potencijalno rešile probleme ili makar olakšale studentima (to bi bile one vaše preporuke).
Meni bi lično takva struktura(naravno, po potrebi bi možda mogli još nešto da ubacite), bila preglednija mnogo. Trenutno vam je sve nabacano, susedne podsekcije nemaju nikakve sličnosti, kad bi izmešali redosled ovih deset podsekcija mislim da niko ne bi ni primetio. Po ovoj mojoj ideji sve bi bilo čitljivije i spregnutije. \odgovor{Slažemo se da je prva sekcija o polu ispitanika suvišna i uklonili smo je iz teksta, a podatke koji su bili prikazani u njoj smo ubacili u okviru teksta koji je sad uvodni deo Sekcije 2. Što se tiče strukture našeg rada; mislimo da je vaša ideja ,da se analiza i diskusija raydvoje u zasebne sekcije, sasvim na mestu ali takođe smatramo da je trenutna struktura rada sasvim u redu i da je nije potrebno drastično menjati. }

\section{Sitne primedbe}
% Напишете своја запажања на тему штампарских-стилских-језичких грешки
U uvodu bih umesto "integracija u akademski sistem" radije iskoristio prilagođavanje na akademski sistem. \odgovor{Deo teksta koji je sadržao to je svakako izmešten.} \\
Na početku druge sekcije imate grešku u kucanju (zaključci). \odgovor{Ispravljeno.} \\
U sekciji 2.1 i u svim drugim imamo fraze "Zaključak na osnovu slike/tabele N:" koje zvuče previše direktno i neformalno a takođe su i boldovane, što je nekorektno. \odgovor{Kao što je odgovorneo na prethodne primedbe, ovo je ispravljeno.} \\
U zaključku imate grešku u kucanju "gorepomenute". \odgovor{Sa našim znanjem iz pravopisa mislimo da ovo nije greška. Prema srpskom pravopisu pravilno je i gorepomenute i gore pomenute. Gorepomenute se piše spojeno kada se prilog gore posebno ne naglašava. Kada gore želi da se naglasi, onda se u tim slučajevima piše odvojeno. Primeri u rečenicama: Da li je gore pomenuto sve šta treba da se kupi?}\\



\section{Provera sadržajnosti i forme seminarskog rada}
% Oдговорите на следећа питања --- уз сваки одговор дати и образложење

\begin{enumerate}
\item Da li rad dobro odgovara na zadatu temu?\\
Ne, trenutna tema ne odgovara sadržaju rada po mom mišljenju. Promenio bih naslov u Kako olakšati studentima proces studiranja ili nešto slično. \odgovor{Naslov rada je promenjen u 'Analiza izazova i olakšica za studente prve godine Informatike na  Matematičkom fakultetu'.}
\item Da li je nešto važno propušteno?\\
Mislim da naslov ne odgovara radu, treba ga promeniti. Pored toga mislim da treba promeniti strukturu projekta, popraviti sažetak i skratiti uvod. \odgovor{Kao što je prethodno napomenuto naslov je promenjen, a sažetak je popravljen i uvod skraćen. Smatramo da strukturu projekta nije potrebno menjati, detaljnije o tome je u odgovoru na poslednju primedbu na strani 8.}
\item Da li ima suštinskih grešaka i propusta?\\
Da, naslov, loša struktura, mnogo tehničkih grešaka oko plotova, tabela, referenci. \odgovor{Nalov, tabele i reference su promenjeni, jedini plot koji je bio u tekstu sada je uklonjen, struktura smatramo da nije loša.} 
\item Da li je naslov rada dobro izabran? \\
Mislim da nije, više su govorili o tome kako olakšati studentima studije nego o tome šta je najlakše studentima prve godine. Možda bi naslov trebao da bude Kako olakšati/poboljšati studije? \odgovor{Nalov je promenjen i smatramo da je sada prikladniji za ono što se obrađuje u ovom radu.} 
\item Da li sažetak sadrži prave podatke o radu?\\
Da, ali mislim da je prvi deo sažetka bespotreban i da umesto toga mogu dodati jos jednu/dve rečenicu koja govori o samom radu. \odgovor{Prvi deo sažetka je izmenjen ali nismo dodali ništa više zato što smatramo da je dovoljno o radu već rečeno unutar njega.} 
\item Da li je rad lak-težak za čitanje?\\
Lak za čitanje poprilično ali ima dosta nepovezanih stvari, svaka sekcija je potpuno nepovezana sa drugim. Bilo bi puno lakše da je sve povezano i da se tokom čitanja možemo usresrediti i samo pratiti flow istraživanja. \odgovor{Nemamo ništa konkretno da odgovorimo na ovo sem da je to više stvar lične preference.} 
\item Da li je za razumevanje teksta potrebno predznanje i u kolikoj meri?\\
Nije potrebno nikakvo predznanje, potrebno je samo razumeti osnovne pojmove vezane za fakultet (šta su konsultacije, prijemni...).
\item Da li je u radu navedena odgovarajuća literatura?\\
Jeste. Samo link ka prvoj knjizi na koju je referisano ne radi. \odgovor{Link je ispravljen.} 
\item Da li su u radu reference korektno navedene?\\
Jesu korektno navedene ali nisu funkcionalne(nije moguće kliknuti na njih), verovatno se provukla neka sitna tehnička greška \odgovor{Reference su popravljene i sada su funkcionalne.} 
\item Da li je struktura rada adekvatna?\\
Po mom mišljenju nije. Sekcija za sekcijom predstavlja analizu na nepovezana i deluje totalno random nabacana pitanja. Predlažem da imate dve velike sekcije, u prvoj ćete vršiti analizu rezultata ankete i diskutovati o tome šta predstavlja problem studentima, koji su problemi najveći i tako slične stvari. U drugoj sekciji možete diskutovati o efikasnom rešavanju tih problema, kako ih rešavaju na drugim boljim univerzitetima itd. Na ovaj način sadržaj sekcija će biti spregnut, lakši za čitanje i biće logički povezan. \odgovor{Razumemo primedbu, ali smatramo da je struktura sasvim dobra i da nisu potrebne drastične promene.} 
\item Da li rad sadrži sve elemente propisane uslovom seminarskog rada (slike, tabele, broj strana...)?\\
Nema nijedne slike, samo grafikon i tabele. Nije ispoštovano to da sadržaj mora biti na prvoj strani. \odgovor{Sadržaj je premešten na prvu stranu.} 
\item Da li su slike i tabele funkcionalne i adekvatne?\\
Slike nema. Tabele i grafikon nisu funkcionalni, ispadaju iz okvira magine, nije referisano na njih, naslovi tabela nisu iznad nego ispod. Sadržaj tabela je koristan i dobar tako da mogu reći da su adekvatne. \odgovor{Grafikon je uklonjen kao dao odgovora i popravke na jednu od prethodnih primedbi. Svi pomenuti problemi sa tabelama su rešeni.}
\end{enumerate}

\section{Ocenite sebe}
Napišite koliko ste upućeni u oblast koju recenzirate: \\
a) ekspert u datoj oblasti\\
b) veoma upućeni u oblast\\
c) srednje upućeni\\
d) malo upućeni \\
e) skoro neupućeni\\
f) potpuno neupućeni\\
Obrazložite svoju odluku\\
b) Kao neko ko je bio student fakulteta bez ikakvog predznanja o programiranju i bez starijih poznanika koji bi mogli da pruže smernice, iz ličnog iskustva mogu da kažem da vrlo dobro razumem izazove i olakšice sa kojima se studenti suočavaju. Znam šta je meni i mojim kolegama bilo teško, a šta nam je olakšavalo prilagođavanje, što mi daje jedinstvenu perspektivu u analizi ove teme.


\chapter{Recenzent \odgovor{--- ocena: 4} }


\section{O čemu rad govori?}
% Напишете један кратак пасус у којим ћете својим речима препричати суштину рада (и тиме показати да сте рад пажљиво прочитали и разумели). Обим од 200 до 400 карактера.

U radu se govori o procesu studiranja, specifično procesu studiranja studenata Matematičkog fakulteta na prvoj godini na smeru Informatika. Istražuju se poteškoće studenata tokom studiranja. U radu se prezentuju rezultati ankete sprovedene nad studentima, kao i smernice za unapređenje predavanja.

\section{Krupne primedbe i sugestije}
% Напишете своја запажања и конструктивне идеје шта у раду недостаје и шта би требало да се промени-измени-дода-одузме да би рад био квалитетнији.

Postavljen je veliki akcenat na poteškoće studenata, što samo po sebi nije loše. Loše je zato što nije izveden zaključak šta je to najlakše studentima prve godine. Ovo bi se moglo popraviti promenom sa obrade negativnih odgovora, na obradu pozitivnih odgovora. Istraživanje vođeno pitanjima poput "Na šta su studenti odgovorili sa 'lako'?" pozitivno bi doprinela radu. Takođe, zaključak bi trebalo da jasno kaže šta su ispitanici u ovom radu označili za 'lako' i šta je potencijalno najlakše. \odgovor{Naslov rada je promenjen u 'Analiza izazova i olakšica za studente prve godine Informatike na  Matematičkom fakultetu', tako da smatramo da rad sada bolje odgovra na to, kao i da je zaključak prikladniji.}


Mislim da treba biti oprezan kada se daju preporuke i saveti u naučnim radovima. Ovaj rad sadrži dosta preporuka, a ne vidi se da su izvedene iz podataka ili referenci. Ne znam da li treba da se ostavi još komentara na tu temu. \odgovor{Slažemo se sa time da treba biti oprezan sa davanjem preporuka i saveta u naučnim radovima. S tim u vidu oprezno smo razmatrali koje preporuke da navedemo. Naše preporuke nisu izvedene direktno iz konkretnih podataka ili referenci već smo sami, grupnim razmišljanjem, došli do zaključaka šta mislimo da bi bila neka od najboljih rešenja za probleme navedene u radu.}
\hfill \break

\section{Sitne primedbe}
% Напишете своја запажања на тему штампарских-стилских-језичких грешки

U sažetku, umesto "prikupljanje informacija", treba "prikupljanja informacija". \odgovor{Prihvaćeno i izmenjeno.} \hfill \break
Tabela 2 sadrži grešku, umesto "Pedznanje", treba "Predznanje". \odgovor{Prihvaćeno i izmenjeno.}\hfill \break
U sekciji 2.7, umesto "nije koristila", treba "nisu koristili". \odgovor{Prihvaćeno i izmenjeno.}\hfill \break
U Tabeli 9, jedna od kolona ima grešku, umesto "Najčeće", treba "Najčešće". \odgovor{Prihvaćeno i izmenjeno.}\hfill \break
\break
Rečenica u uvodu: "Osim toga, istraživanja pokazuju da ... što bi im pomoglo da prevazidu izazove na početku studija." je predugačka i teška za razumevanje. Trebalo bi je podeliti na više rečenica, ili promeniti strukturu na drugi način. \odgovor{Struktura je promenjena na drugi način, uvideli smo da celom pasusu nije mesto tu i da zauzima previše mesta u uvodu pa smo ga izmestili na početak drugog dela rada.}\hfill \break
\break
U radu je skoro uvek korišćen pasiv nasuprot aktivu. Na mestima poput "fokusiramo se"(sažetak), "\vspace{0cm}Smatramo da"(uvod) i "istraživanja pokazuju"(zaključak) aktiv ometa pažnju. \odgovor{Posle detaljnijeg čitanja uvažavamo primedbu i izmenili smo takve delove teksta} \hfill \break
\break
Reference nisu "\vspace{0cm}clickable", a trebalo bi da budu. \odgovor{Ispravljeno.} \hfill \break
Prva referenca ne vodi ka izvoru, link je pokvaren. \odgovor{Postavljen je ispravan link.} \hfill \break


\section{Provera sadržajnosti i forme seminarskog rada}
% Oдговорите на следећа питања --- уз сваки одговор дати и образложење

\begin{enumerate}
\item Da li rad dobro odgovara na zadatu temu?\\
Donekle. Najveći deo rada je posvećen stvarima koje su studentima teške i kako ih rešiti i taj deo teme je dobro obrađen. \hfill \break

Sa druge strane, iz rezultata ankete su izvučeni i obrađeni negativni odgovori. Preporuke koje slede iza pitanja savetodavne su prirode i nisu zasnovane na anketi ili vezane za glavnu temu.
%(na primer: Studenti su davali odgovore na pitanje o predznanju iz relevantne oblasti. Preporuke koje slede nisu proizašle iz propratnih pitanja. Da li je studentima problem što nemaju predznanje pre upisa na fakultet? Da li bi se studenti složili sa preporukama? Da li postoji relevantna literatura koja pokazuje da bi?)%
\hfill \break

Sve smernice što se tiče opisa teme su ispoštovane. \\
\odgovor{Promenili smo naslov teme, pa smatramo da bi sada rad trebalo dobro da odgovra na temu.}
\item Da li je nešto važno propušteno?\\
Da. Pozitivni aspekti studiranja na prvoj godini. Iako postoji jasna slika šta je studentima teško, ne vidi se šta im je lako. \odgovor{Smatramo da je promenom naslova tema adekvatno obrađena u radu kao i da samim tim ništa važno nije propušteno.}

\item Da li ima suštinskih grešaka i propusta?\\
Pomalo. Ne postoji kratak opis sadržaja ankete. Tabele prikazuju samo deo odgovora na pitanje (par kategorija), što može stvarati probleme kod interpretacije. Metodologija kojom su postignuti zaključci na odgovore na pitanja nije korektna. \odgovor{Tabele sadrže samo deo odgovora(uglavnom najpozitivnije ili najnegativnije) da bi pokazale da li je odredjena tema/pitanje studentima lako ili teško. Smatramo da ostatak odgovora u tabeli ne bi previše doprineo, kao i da je metodologija kojim su postignuti zaključci korektna.}

\item Da li je naslov rada dobro izabran?\\
Ne. Sadržaj rada je bolje opisan sa naslovom "Šta je teško studentima prve godine". \odgovor{Naslov je promenjen u 'Analiza izazova i olakšica za studente prve godine Informatike na  Matematičkom fakultetu'}

\item Da li sažetak sadrži prave podatke o radu?\\
Da. Sažetak dobro prikazuje šta se izučava u radu.

\item Da li je rad lak-težak za čitanje?\\
Lak. Postoji dobra ravnoteža između slika/tabela i teksta, rečenice su smislene. Tekst je dobro podeljen i ostavljeno je dovoljno mesta između stavki. 

\item Da li je za razumevanje teksta potrebno predznanje i u kolikoj meri?\\
Nije. Tekst ne koristi nepoznate reči i objašnjava koncepte na razuman način.

\item Da li je u radu navedena odgovarajuća literatura?\\
Donekle. Osim linka koji nedostaje, ostali su odgovarajući. \odgovor{Stavljen je odgovarajući link.}

\item Da li su u radu reference korektno navedene?\\
Jesu, mada nisu povezane za sekciju "Literatura" (ne mogu se kliknuti). \odgovor{Reference su popravljene i sada su ispravne.}

\item Da li je struktura rada adekvatna?\\
Da. Radom preovladava obrada ankete i davanje smernica. Postoje jasno odvojeni uvod, razrada i zaključak.

\item Da li rad sadrži sve elemente propisane uslovom seminarskog rada (slike, tabele, broj strana...)?\\
Da. Svi elementi su prisutni i u dobrom broju.

\item Da li su slike i tabele funkcionalne i adekvatne?\\
Ne skroz. Tabelama nedostaje deo informacija. \odgovor{Smatramo da tabele sadrže sve informacije koje su potrebne za prikaz informacija na određeno pitanje.}
\end{enumerate}

\section{Ocenite sebe}
% Napišite koliko ste upućeni u oblast koju recenzirate: 
% a) ekspert u datoj oblasti
% b) veoma upućeni u oblast
% c) srednje upućeni
% d) malo upućeni 
% e) skoro neupućeni
% f) potpuno neupućeni
% Obrazložite svoju odluku
Malo sam upućen u temu. Na raspolaganju su mi moje i iskustvno bližih ljudi. Van toga, nisam se mnogo upućivao u temu.


\chapter{Dodatne izmene}
%Ovde navedite ukoliko ima izmena koje ste uradili a koje vam recenzenti nisu tražili. 

\end{document}
